% !TEX TS-program = xelatex
% !TEX encoding = UTF-8 Unicode
% !Mode:: "TeX:UTF-8"

\documentclass{resume}
\usepackage{zh_CN-Adobefonts_external} % Simplified Chinese Support using external fonts (./fonts/zh_CN-Adobe/)
% \usepackage{NotoSansSC_external}
% \usepackage{NotoSerifCJKsc_external}
% \usepackage{zh_CN-Adobefonts_internal} % Simplified Chinese Support using system fonts
\usepackage{linespacing_fix} % disable extra space before next section
\usepackage{cite}

\begin{document}
\pagenumbering{gobble} % suppress displaying page number

\name{王海蔚}

\basicInfo{
  \email{wanghaiwe.i.1999@163.com} \textperiodcentered\ 
  \phone{(+86) 182-6225-8003} 
}
 
\section{\faGraduationCap\  教育背景}
\datedsubsection{\textbf{182-6225-8003}, 182-6225-8003}{2017 -- 至今}
\textit{在读本科生}\ 软件学院, 数字媒体技术, 预计 2021 年 7 月毕业

\datedsubsection{\textbf{东北大学理学院大学数学网络考试平台选考系统重制工程}}{2018年6月 - 2020年6月}
\role{Web/Vue.js/kotlin}{团队项目-前端/运维}
该项目库为私有库,如需查看代码请联系本人
\begin{onehalfspacing}
项目地址: https://github.com/wanghaiwei/xkxt-reloaded-frontend
简介:原有的东北大学理学院大学数学网络考试平台选考系统重构项目。技术上前后端分离,用Postgresql作为数据库,kotlin为后端编程语言,采用Vert.x高性能框架提供Api服务,静态页面采用Vue.js框架。保证静态页面服务不会断,以提高用户体验。
\newline
前端特性
\begin{itemize}
  \item 放弃原有的Spring 框架,采用了前后端分离的架构
  \item 采用了Vue.js框架,组件库使用了Muse-ui和iview等UI框架
\end{itemize}
\end{onehalfspacing}

\datedsubsection{\textbf{基于修罗/xiunoBBS的docker方案}}{2019年6月}
\role{Docker}{个人项目}
\begin{onehalfspacing}
一款开源PHP论坛xiunobbs的Docker化方案, https://github.com/wanghaiwei/xiuno-docker
\begin{itemize}
  \item 可通过配置文件快速部署
  \item 方便运行环境隔离
  \item 简化常规安装的繁琐操作
\end{itemize}
\end{onehalfspacing}

\datedsubsection{\textbf{《舞侠》项目}}{2019年7月– 2020年2月}
\role{Web/Vue.js}{团队项目-前端}
该项目库为私有库,如需查看代码请联系本人
\begin{onehalfspacing}
项目地址:https://github.com/wanghaiwei/DanceKnight-frontend
\newline
简介:一个舞蹈类社区项目。目标是仿知乎一样的大型领域专业类问答社区,同时包含信息发布浏览系统。
\end{onehalfspacing}

\section{\faCogs\ IT 技能}
% increase linespacing [parsep=0.5ex]
\begin{itemize}[parsep=0.5ex]
  \item 编程语言: javascript/C\#/C++
  \item 平台: Linux/Windows
\end{itemize}

\section{\faHeartO\ 获奖情况}
\datedline{\textit{辽宁省二等奖}, 第九届蓝桥杯}{2018年3月}

\section{\faInfo\ 其他}
% increase linespacing [parsep=0.5ex]
\begin{itemize}[parsep=0.5ex]
  \item 技术博客: http://rayfalling.com
  \item GitHub: https://github.com/wanghaiwei
  \item 自建Git: https://git.rayfalling.com/rayfalling
\end{itemize}

%% Reference
%\newpage
%\bibliographystyle{IEEETran}
%\bibliography{mycite}
\end{document}
